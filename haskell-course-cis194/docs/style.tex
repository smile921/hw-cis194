\documentclass[12pt]{article}

\newcommand{\doo}{\textbf{DO}}
\newcommand{\dont}{\textbf{DON'T}}

\pagestyle{empty}

\begin{document}

\begin{center}
{\LARGE Good Haskell Style}
\end{center}
\bigskip

All your submitted programming assignments should \emph{emerge
  creatively} from the following style guidelines.  Programming is
just as much social art form as it is engineering discipline, and as
any artist knows, constraints serve to enhance rather than quench
creativity.

These guidelines will be extended as the semester progresses.

\begin{itemize}
\item \doo\ use \texttt{camelCase} for function and variable names.

\item \doo\ use descriptive function names, which are as long as they
  need to be but no longer than they have to be.  Good:
  \texttt{solveRemaining}.  Bad: \texttt{slv}. Ugly:
  \texttt{solveAllTheCasesWhichWeHaven'tYetProcessed}.

\item \dont\ use tab characters.  Haskell is layout-sensitive and tabs
  Mess Everything Up.  I don't care how you feel about tabs when
  coding in other languages.  Just trust me on this one.  Note this
  does not mean you need to hit space a zillion times to indent each
  line; your Favorite Editor ought to support auto-indentation using
  spaces instead of tabs.

\item \doo\ try to keep every line under 80 characters.  This isn't a
  hard and fast rule, but code that is line-wrapped by an editor looks
  horrible.

\item \doo\ give every top-level function a type signature.  Type
  signatures enhance documentation, clarify thinking, and provide
  nesting sites for endangered bird species.  Top-level type signatures
  also result in better error messages. With no type signatures, type
  errors tend to show up far from where the real problem is; explicit
  type signatures help localize type errors.

  Locally defined functions and constants (part of a \texttt{let}
  expression or \texttt{where} clause) do not need type signatures,
  but adding them doesn't hurt (in particular, the argument above
  about localizing type errors still applies).

\item \doo\ precede every top-level function by a comment explaining
  what it does.

\item \doo\ use \texttt{-Wall}.  Either pass \texttt{-Wall} to
  \texttt{ghc} on the command line, or (easier) put
\begin{verbatim}
{-# OPTIONS_GHC -Wall #-}
\end{verbatim}
at the top of your \texttt{.hs} file.  All your submitted programs
should compile with no warnings.

\item \doo, as much as possible, break up your programs into small
  functions that do one thing, and compose them to create more complex
  functions.

\item \doo\ make all your functions \emph{total}.  That is, they
  should give sensible results (and not crash) for every input.
\end{itemize}

\end{document}
